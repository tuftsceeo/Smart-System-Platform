\cleardoublepage%
\chapter{\label{chap:fut}Future Work}%

This thesis established the Smart System Platform as a proof of concept, demonstrating its potential as a tool for networking and interactive system development. However, future work could expand the scope beyond just a technical prototype of the networking capability and its support framework. Some areas of interest and key question to explore are listed below:

\begin{itemize}
    \item \textbf{Continued development of SSP-based projects and activities}\\\
    Of the many ideas and concepts proposed using the Networking Library during the two hackathons, which could be developed into a game or educational experience? How could a collaborative educational experience be developed using a network? 
    
    \item \textbf{Developing educational applications}\\
    How can the Smart System Platform and its developed networking capabilities be used to improve usability and accessibility, and what novel functionalities or educational opportunities could be enabled by this enhanced connectivity? It would be interesting to develop a lesson plan based on the Smart Motors and their networking capability and explore how such a networking approach could be integrated into science education?
    
    \item \textbf{Further development of software and tools}\\
    As outlined in chapter \ref{chap:res}, there are several areas for improvement, especially in the area of development and management tools. Building on the basic concept, how could the tools be designed to be intuitive and accessible? For example, a Lab-View inspired drag-and-drop graphical user interface could be developed for wireless configuration of modules and their interactions. \\
    Furthermore, would it be possible to fully integrate an LLM to allow full AI-driven configuration of the various modules, based on natural text or image input only? This could greatly simplify the setup process and make the system more accessible to non-technical users. 

    \item \textbf{Integration with other Technologies}\\\
    How is it possible to connect your phone to the designed network? Could we use the phone's sensors to feed data to our modules? In addition, an app version of some tools could be developed to program, configure and interact with the system from a mobile device, further increasing accessibility.\\
    What other technologies could the system be integrated with?
    
    \item \textbf{Hardware and component support development}\\
    Based on the explored possibilities of sensors and outputs, the development of support libraries using our hardware could be continued. We could also explore what other modules could be developed with all the available capabilities.\
    And in a second step, based on the modules developed, what could a networked educational robotics kit look like? What components could be used and what could be avoided? 
    
    \item \textbf{Enlarging the user base}\\
    While the MVP was primarily designed to support the CEEO and educational researchers, the system could be expanded and made widely available to other stakeholders such as teachers and students. Could the platform be integrated with external platforms such as data analysis tools or educational software? What would need to be developed to make it accessible and useful to teachers? How could educators quickly develop and deploy custom or off-the-shelf systems to support STEM education?
    
    \item \textbf{Community building \& open source development}\\
    How can an open source community be fostered to support ongoing development? What strategies would encourage contributions from students, teachers and developers? What tools or platforms would best facilitate collaboration and knowledge sharing? 
\end{itemize}


