\cleardoublepage%
\chapter{\label{chap:fut}Future Work}%

The Smart System Platform, as a proof of concept, was established and introduced in this thesis, demonstrating its potential as a tool for networking and interactive system development. However, future work could extend beyond a technical prototype, further developing the networking capability and its support framework. The following areas of interest and key questions were identified for further exploration:

\begin{itemize}
    \item \textbf{Continued development of SSP-based projects and activities}\\\
    Of the numerous ideas and concepts proposed using the Networking Library during the two hackathons, which could be developed into a game or educational experience? Furthermore, how could a collaborative educational experience be developed, fully leveraging the established networking approach? 
    
    \item \textbf{Developing educational applications}\\
    How can the Smart System Platform and its developed networking capabilities be used to improve usability and accessibility to educational robotics, and what novel functionalities or educational opportunities could be enabled by this enhanced connectivity? It would be interesting to develop a lesson plan centred on the Smart Motors and their networking capability, and to explore how such a networking approach could be integrated into science education.
    
    \item \textbf{Further development of software and tools}\\
    As outlined in Chapter \ref{chap:res}, there are several areas for improvement, especially in the area of development and management tools. 
    The Networking Library and the SSP Add-on Library could be optimised and rewritten in C++. Furthermore, building on the basic concept, how could the tools be designed to be intuitive and accessible? For example, a LabVIEW inspired drag-and-drop GUI could be developed for wireless configuration of modules and their interactions. \\
    Additionally, would it be feasible to fully integrate an LLM to enable full AI-driven configuration of the various modules, based on natural text or image input only? This could greatly simplify and streamline the configuration process and enhance the accessibility for non-technical users. 

    \item \textbf{Integration with other Technologies}\\\
    How is it possible to connect a mobile phone to the designed network? Could the phone's sensors be used to provide data input to the modules? In addition, an app version of some tools could be developed to program, configure and interact with the system from a mobile device, further increasing accessibility.\\
    What other technologies could the system be integrated with?
    
    \item \textbf{Hardware and component support development}\\
    In light of the explored possibilities of sensors and outputs, the development of support libraries using the hardware could be continued. Furthermore, the development of additional modules and capabilities could be pursued and the integration or use of novel hardware explored.
    And in a second step, based on the modules developed, what could a networked educational robotics kit look like? What components could be further used? 
    
    \item \textbf{Enlarging the user base}\\
    Although the MVP was principally designed to support the CEEO and educational researchers, the system could be expanded and made widely available to other stakeholders such as teachers and students. How could the platform be integrated with external platforms such as data analysis tools or educational software? In order to facilitate its accessibility and utility for teachers, what further developments would be necessary? How could educators quickly develop and deploy custom or off-the-shelf systems to support STEM education?
    
    \item \textbf{Community building \& open source development}\\
    How can an open source community be fostered to support ongoing development? What strategies could be implemented to encourage contributions from students, teachers and developers? What tools or platforms would most effectively facilitate collaboration and sharing of knowledge? 
\end{itemize}


