\cleardoublepage%
\chapter{\label{chap:intro}Introduction}%

\section{\label{sec:intro_background_motivation}Background and Motivation}%

\subsection{\label{sec:intro_se}Science, STEM and STEAM Education}%
Science education has become an integral component of modern general education \citep{atkin_inside_2003}, and, according to the \citet{national_research_council_national_1995, national_research_foundation_science_2022}, is widely acknowledged as crucial for cultivating a scientifically literate population, capable of comprehending the governing principles of our universe and engaging with our increasingly intricate science- and technology-driven world \citep{bybee_case_2013}. A scientifically literate populace further promotes the contribution to national progress in a variety of areas, such as in economic and scientific development. \citep{atkin_inside_2003}

\begin{quote}
    "Science is more than a body of knowledge; it is a way of thinking." - Carl Sagan \citep{sagan_demon-haunted_1995} 
\end{quote}

Historically, the primary focus of K-12 science education has been the imparting of knowledge in various scientific subjects, such as biology, chemistry, maths, and physics, in an isolated and purely factual manner. However, in recent times, there has been an increasing emphasis on cultivating critical thinking skills, problem-solving abilities, and the capacity to discern the interconnected relationships between diverse scientific disciplines. \citep{national_research_council_national_1995, noauthor_next_2013, atkin_inside_2003} 

\begin{quote}
    "The principle goal of education in the schools should be creating men and women who are capable of doing new things, not simply repeating what other generations have done; men and women who are creative, inventive and discoverers, who can be critical and verify, and not accept, everything they are offered." - Jean Piaget \citep{duckworth_piaget_1964}
\end{quote}

Examples of this line of thinking are initiatives such as STEM (Science, Technology, Engineering, and Mathematics) education, which build on top of general science education. STEM education goes beyond teaching scientific subjects in an isolated fashion, instead emphasises interdisciplinary applications and real-world problem-solving, as outlined by \citet{bybee_case_2013, xie_stem_2015, blackley_stem_2015, abdi_tracing_2024}. \citet{deville_stem_2024} further states that STEM is fundamentally connected to everything in our society, and that science education must help students see those connections. More recently, STEAM (Science, Technology, Engineering, Arts, and Mathematics) education has aimed to take STEM a step further by also including the arts. This holistic approach further focuses on the importance of creativity and design thinking in innovation, and the pedagogy thereof \citep{marin-marin_steam_2021, dolgopolovas_computational_2021, connor_stem_2015, bequette_place_2012}. Research, such as that by \citet{jamali_role_2023, yakman_exploring_2012, samsudin_effect_2020} has shown that both STEM and STEAM approaches can significantly enhance learning, as measured by an improvement in soft skills such as creativity, critical thinking, problem-solving, collaboration and communication, and create an increased interest in STEM fields, and as such can be said to improve the quality of education.

\subsection{\label{sec:intro_et}Education Technology and SDG 4}%

Educational frameworks, methods, and technology have been constantly evolving, as seen by the aforementioned STEM and STEAM education approaches and reflected in the Next Generation Science Standards (NGSS) \citep{noauthor_next_2013}. In terms of methodology, there has been a shift from basic factual learning to more engaging and interactive approaches \citep{atkin_inside_2003, noauthor_next_2013, }. Promising examples of such approaches include hands-on learning \citep{satterthwait_why_2010, vesilind_hands-_1996}, project-based learning \citep{sawyer_project-based_2014, kokotsaki_project-based_2016, samsudin_effect_2020, markula_key_2022}, learning through play \citep{weisberg_guided_2013, zosh_learning_2017, parker_learning_2022}, as well as group-based collaborative learning \citep{tonkal_exploring_2024, brennan_implementing_2023}, which have all shown to have a positive impact on learning, and consequently on quality of education. Engineering technology, and educational robotics in particular, have played a major role in the development and implementation of many of these methods, as outlined by \citet{khine_robotics_2017}. In the context of educational robotics, research has generally demonstrated an enhancement in learning outcomes, especially in the domain of STEM-related concepts. \citep{benitti_exploring_2012, lego_education_stem_nodate, afari_robotics_2017}. 
\\\\
As described by \citet{sapounidis_educational_2020}, educational robotics first emerged out of Papert's theory of learning called constructionism \citep{papert_constructionism_1991}, based on Piaget’s ideas on constructivism \citep{von_glasersfeld_interpretation_1982}, which can be summarised as learning-by-making and connects the concepts of learning and play. This formed the basis for further educational robotics technology, such 
as the LEGO Mindstorms Robotic Invention Kit \citep{mindell_lego_2000}, which was developed in cooperation by the MIT Media and the LEGO Group's education division (LEGO Education, formerly LEGO Dacta).
A more recent example is \citet{noauthor_lego_nodate}, an NGSS-based K-8 hands-on science curriculum that provides both lesson plans and the educational technology, in form of LEGO bricks and a new educational robotics kit, to support science education. \citep{lego_education_stem_nodate} 
\\\\
Such initiatives align closely with the United Nations Sustainable Development Goals (SDGs) \citep{united_nations_department_of_economic_and_social_affairs_17_nodate}, specifically SDG 4 - Quality Education \citep{united_nations_department_of_economic_and_social_affairs_goal_nodate}, which is part of the 17 SDGs  adopted by all UN member states in 2015 as part of the 2030 Agenda for Sustainable Development \citet{united_nations_department_of_economics_and_social_affairs_sustainable_nodate}. The International Society for Technology in Education (ISTE) has also developed standards that align with SDG 4, which further emphasises the role of technology in achieving quality education for all \citep{iste_iste_nodate, noauthor_technology_2023}.
\\\\
However, while the emergence of educational initiatives, such as STEM and STEAM, as well as the implementation of new methods and technologies, such as educational robotics, have resulted in advancement in educational quality, there still remains considerable work to be done to ensure the widespread accessibility and inclusivity of these technologies for all \citep{sapounidis_educational_2020}, in accordance with all tenets of SDG 4. Particularly in the case of educational robotics. Accessibility factors such as cost, usability and high entry barriers pose challenges, as stated by \citet{dahal_designing_2024, dahal_international_2023, johnson_implementation_2012}.

\subsection{\label{sec:intro_smp}The Smart Motors Project}%
In response to the recognised need for affordable robotic technologies in the educational sector \citep{khine_robotics_2017}, Smart Motors, a low-cost, open-source solution for educational robotics, has been developed by \citet{dahal_designing_2024} at the Tufts Center for Engineering Education and Outreach (CEEO). As the name suggests, the concept is centred around a motor, though other forms of outputs, such as visual, audio, haptic etc., could also be utilised. A further component is the user interface, enabling interaction with the motor, which in the latest version of the motor includes a screen, three buttons and a potentiometer. Another component of the system is the input, in the latest version of the Smart Motor this is either a built in accelerometer or some other analogue sensor that can be plugged into the system. All these components are connected to and by a microcontroller.
The Smart Motor is a self-contained, trainable educational robotics system for STEM classrooms. It has been designed to be trained and programmed directly on the unit itself, thereby eliminating the need for additional software and hardware to program the motor. This makes it intuitively usable out of the box without the need for any other supporting software or hardware, and independent of any infrastructure, with the exception of electricity to charge the battery. Its design is intended to be as accessible and intuitive as possible, with the objective of reducing its acquisition and production cost and lowering the entry barriers for use. The motor has two modes of operation, in the training mode, the user can program and train the motor output based on certain sensor input. The result of this training is then used to determine the relationship between input and output in the second mode, the play mode. \citet[][p. 3]{dahal_designing_2024} characterises them as a "Trainable Motor for Storytelling: in response to the challenge of introducing robotics to a classroom, especially with limited access to computers and Wi-Fi".
The Smart Motor was designed for Elementary and Middle School students (K-8). The primary teaching objective being the instruction of reinforcement learning to help teach Machine Learning (ML) and Artificial Intelligence (AI) using a hands-on and playful learning approach enabled by the Smart Motors. In the current approach, each motor is used as an individual self-contained unit. Though given the incorporation of general-purpose components and open-source firmware and software, the system possesses a plethora of inherent capabilities, including Bluetooth and Wi-Fi networking, and support for various other inputs and outputs. Furthermore, with a microcontroller at its core, the system can be easily reprogrammed to implement different logic or programs, and adapted to the use of various additional hardware components, among other advantages. This versatility underscores the system's considerable potential and the prospect of expanding its range of applications. \citep{dahal_designing_2024}

\section{\label{sec:intro_res_quest}Justification, Goals and Research Questions}

The potential and untapped capabilities of Smart Motors are explored, particularly in the context of implementing a networking approach that facilitates communication and interaction between Smart Motors and other modules. The aim is to develop a networking approach and supporting material, to enable and explore the application of these modules for educational projects and the ways in which educational researchers can use these technologies to develop new innovative ideas for hands-on learning, learning through play and collaborative learning. The main focus of the thesis lies on the networking, tool and system development and testing.
\\\\
In this thesis the Smart Systems Platform (SSP) is introduced, a networking-enabled engineering education platform, based on Smart Motors.
With the development of networking at its core, the platform is designed to support, enable and make this capability accessible for easy use and development. 
The platform is a minimum viable product (MVP) that demonstrates the Smart Motors capabilities and how they can be leveraged into a more broad and general educational robotics system. The platform's overarching system design encompasses a platform architecture approach, which considers the overall system design, nomenclature, components (hardware and software), supporting systems and documentation, usability in form of the of development tools and management suite, and more to facilitate interaction, usability and further reduce entry barriers. The platform also includes a roadmap for future work, while considering the core motivation of the Smart Motors project of being a low cost and accessible and open source educational resource. The advantages of such a system could be numerous. It would serve to simplify the development of new educational projects and approaches by researchers and educators. Furthermore, it could play a pivotal role in enhancing the quality and accessibility of education. This is due to its potential to function as a foundation for an educational robotics kit and as a pedagogical instrument in the classroom, where the networking capabilities specifically might enable new and novel approaches. Additionally, it would contribute to general accessibility on an institutional or regional level in terms of cost and sourcing, thanks to its low-cost and open-source nature. 
\\\\
The research questions are:

\begin{enumerate}
    \item How can we design and implement a peer-to-peer networking approach for Smart Motor-based devices to enable inter-module communication?
    \item How can these networking capabilities be made accessible? What framework or architecture is necessary for a Smart System Platform to support and enable access and effective use of these capabilities?
    \item Given access to this networking capability and its support Smart System Platform framework, how do students and educational researchers utilise it, and what do they come up with?
    
    % \item How can a peer-to-peer networking protocol for Smart Motors be developed to enable inter-module communication?
    % \item How can these Smart Motor-based networking capabilities be made available and usable in an accessible way? What kind of framework or of such a Smart System Platform is needed to support and enable this?
    % \item How can students and educational researchers use such a facility when given access to it and its supporting framework and what do they come up with?
    % \item How can such networking between between modules be enabled and developed?
    % \item How can a networking enabled Smart Motors-based system be developed and supported? What is needed to interact and make use and development using the networking?
    % \item What happens when you give the Smart System Platform to college students and education researchers?
\end{enumerate}

The central objective of the thesis is the development of networking capabilities for Smart Motors. In addition, the support, enablement and ease of use of this capability is examined, which has resulted in the creation and development of the Smart System Platform as a MVP. Furthermore, the developed system and capabilities are then tested with a small focus group of college students, examining how they can use the developed capabilities and platform.

%\begin{itemize}
%    \item Overarching System Platform Architecture \& Road-map, come up with the platform architecture design, identify key capability needs and the underlying goal of the whole thing, as well as design concepts
%    \item Develop specific key capabilities (Website, IDE \& UI, Wi-Fi Mesh, (Hive motors) \& IDE UI, certain firmware for certain component hardware integration)
%    \item Example Robotics Kit for Validation and Showcases
%\end{itemize}


%Main focus on the capability and tool developments, focus on developing and testing the networking and the platform tools to answer the following research questions: First focus on developing the networking between smart motor and other modules. What can networking enable, what hardware could be connected, what projects made a reality? Smart Playground. Developing the Smart System Platform as a whole, designing its architecture and its many components, from hardware and software, to the project webpage and the GitHub repository and last the development and management tools to help use and develop things with the systems networking capabilities. 
%The SSP would consist of various yet to be determined hardware modules, such as the Smart Motor, but would also include the . Furthermore, it would include a website hub, which hosts general information on the SSP, information on sourcing of modules and components, the necessary firmware, assembly instructions, how-to guides, examples, lesson plans and an integrated IDE, aimed at students, teachers and academics, creating a base platform on which future projects and educational tools can be built by the CEEO, as well as educators all over the world: A Smart System Platform in Learning and Play, if you will.
%The ultimate goal of the SSP is to be a tool to develop projects and be in itself a tool that gives its users an intuitive understanding for the science and engineering principles, among others, without going through all the boring theory. In addition we continue to follow the aim that it should lower the entry barrier into, in terms of affordability and complexity, for students, teachers and educators, allowing them to use the system in their own way and also come up with new and inventive ways to teach through hands-on learning, learning through play and collaborative learning!