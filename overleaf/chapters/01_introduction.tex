\cleardoublepage%
\chapter{\label{chap:intro}Introduction}%

Introduce the topic. Education, why is education important etc.

\section{\label{sec:intro_ee}STEM Education}%

Broad overview of Education in Engineering, robotics kits and other main background topics:  what it is, which SDGs are important (SDG 4, Quality Education). , some global statistics, etc.  This is going to prepare the reader for the next sections, which will provide more detail on specific sub-sectors or processes.

Mention limitations of existing systems, be it monetary, usability, entry barriers, find papers that talk about this. Milan thesis.

\section{\label{sec:intro_smp}Smart Motors Project}%

The Smart Motors project has specifically been developed with the aim to teach reinforcement learning. 
What were the focus points of the smart motor project? Modularity, Cost effectiveness, accessibility (both in terms of monetary and management, sourcing etc.). \citep{smartmotors}

Introduce the Smart Motors Project:

The Smart Motors are a trainable educational robotics system for STEM classrooms, which were developed and evaluated as part of a dissertation at the CEEO. They are described as a "Trainable Motor for Storytelling: in response to the challenge of introducing robotics to a classroom, especially with limited access to computers and WiFi"\footnote{}.
It is a programmable, trainable motor that works with various sensor inputs and outputs. Uniquely, the motor is designed to be trained and programmed directly on the device itself, eliminating the need for additional software and hardware to program the motor, so it can be used out of the box!

The Smart Motor is aimed at Elementary and Middle School students, to teach certain robotics and reinforcement learning principles, but offers many more application options! It works similar to other robotic kits, such as the LEGO Spike, but is designed specifically to be as accessible as possible, reducing its acquisition and production cost and lowering the entry barriers for use, among other things.

Quote results of Smart Motors and why there is a use case for it, quote success of similar projects, LEGO Education foremost. 

\section{\label{sec:intro_res_quest}Justification and Research Questions}

In your final paragraph, you need to clearly frame the justification for the study -- this should be obvious at this point from the clear discussion that you have presented thus far. Here, you spell it out in concise terms. Here, or in a separate subsection, you also spell out the specific research questions that you have, along with, if relevant, your hypotheses. The combination of the justification and research questions will be a clear and obvious bridge to the next chapter: Methods.


Goal:
I would like to develop the Smart Motors further, using a broader product and system development approach. The idea would be to come up with a system platform and architecture for Engineering Education, taking the Smart Motors as a basis. Introducing the Smart Systems Platform (SSP) as an engineering education platform. The goal of the thesis would be to come up with the overarching system design of the SSP, its needed capabilities, its architectures and creating a road-map for its development, as well as to develop certain elements directly as part of the thesis. In addition, I would like to come up with the basic robotics kit built and lesson plan built on this system platform as a showcase and for validation purposes.
The SSP would consist of various yet to be determined hardware modules, including but not limited to the Smart Motor itself, Input (Sensor) Modules, Output Modules (Motor, audio, visual, haptic etc.) and other interactive modules. Furthermore, it would include a website hub, which hosts general information on the SSP, information on sourcing of modules and components, the necessary firmware, assembly instructions, how-to guides, examples, lesson plans and an integrated IDE, aimed at students, teachers and academics, creating a base platform on which future projects and educational tools can be built by the CEEO, as well as educators all over the world: A smart system platform in Learning and Play, if you will.
The ultimate goal of the SSP is to be a tool to give its users an intuitive understanding for the used engineering principles, among others, without going through all the boring theory. In addition it should lower the entry barrier into, in terms of affordability and complexity, for students, teachers and educators, allowing them to use the system in their own way and also come up with new and inventive ways to teach through Hands-on learning and discovering through play!


\begin{itemize}
    \item Overarching System Platform Architecture \& Road-map, come up with the platform architecture design, identify key capability needs and the underlying goal of the whole thing, as well as design concepts
    \item Develop specific key capabilities (Website, IDE \& UI, Wi-Fi Mesh, (Hive motors) \& IDE UI, certain firmware for certain component hardware integration)
    \item Example Robotics Kit for Validation and Showcases
\end{itemize}



Research Questions:
\begin{itemize}
    \item Feasibility of a low cost Smart Motors based Education System
    \item Can Smart Motors be developed further into more use cases and whatnot.
    \item Capability Extension and whatnot
    \item Accessibility in terms of project development an ease of use.
    \item Is it possible to test and validate a minimal viable product of the Smart System Platform.
\end{itemize}

