\chapter*{Abstract}%
\markboth{\MakeUppercase{Abstract}}{\MakeUppercase{Abstract}}%
\addcontentsline{toc}{chapter}{Abstract}%
\phantomsection%

In this thesis the development of a peer-to-peer networking approach is explored for Smart Motors, an educational robotics platform, and the creation of a supporting framework called the Smart System Platform. The focus of the research is on the design and development of the ESP-NOW based networking protocol, how to make this networking capability accessible through the supporting Smart System Platform framework, with an emphasis on simplicity, accessibility and modularity, and how students and educational researchers can use this platform and technology. \\\\

The networking approach developed consists of a custom networking library based on ESP-NOW that enables direct communication between any ESP32-based microcontrollers, called Smart Modules. The library introduces several features such as an address book system, a message structure with different types and subtypes, and the ability to send larger data packets. Building on this, a Smart System Platform add-on library has been created to provide specific commands and handlers to control and configure the Smart Modules. To support and enable access to these networking capabilities, the Smart System Platform was designed as an overarching system architecture built around the networking capabilities. Based on this platform, a minimum viable product has been developed in the form of a number of platform components, including hardware component concepts, software, documentation and development support sites such as the GitHub page, guides and website, and a suite of network focused development and management tools, including a custom web-based integrated development environment. \\\\

The developed networking approach and platform were tested using various methods, including robustness tests, experiments with range and received signal strength indicators, and two hackathons with college students to see how they use the capabilities provided and what they can come up with. The developed networking approach has also been used by students and as part of other ongoing research projects, such as the Smart Playground project. The results demonstrate the viability of the peer-to-peer communication system and its potential for educational applications, but also highlight shortcomings in the usability and accessibility of the support tools developed. The research also highlights areas for future development, such as improving the quality and accessibility of support tools and materials. This work contributes to the field of educational robotics by providing a flexible, accessible networking solution and support platform that can be built upon to enhance interactive learning environments. By enabling peer-to-peer communication between educational robotics modules, it supports innovative approaches to STEM education, enabling the development of more complex, collaborative and engaging learning experiences.

