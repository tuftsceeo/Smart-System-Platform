\chapter*{Abstract}%
\markboth{\MakeUppercase{Abstract}}{\MakeUppercase{Abstract}}%
\addcontentsline{toc}{chapter}{Abstract}%
\phantomsection%

In this thesis the design and development of a peer-to-peer networking approach for Smart Motors, an educational robotics platform, is explored. Furthermore, in order to structure and facilitate the approach, a framework called the Smart System Platform is developed. The focus of the research lies on the design and development of the ESP-NOW based networking protocol, how to make this networking capability accessible through the supporting Smart System Platform framework, with an emphasis on simplicity, accessibility and modularity, and how students and educational researchers can use this platform and technology. \\

The networking approach developed consists of a custom networking library based on ESP-NOW that enables direct communication between any ESP32-based microcontrollers, which are referred to as Smart Modules. The library introduces several features, including an address book system, a message structure with different types and subtypes, and the ability to send larger data packets. Building on this, a custom Smart System Platform add-on library has been created to provide specific commands and handlers to control and configure the Smart Modules. 
The Smart System Platform has been designed to support and enable access to these networking capabilities, serving as an overarching educational robotics system architecture built around the networking capabilities. Based on this platform, a minimum viable product has been developed in the form of a number of platform components, including hardware component concepts, software, documentation and development support sites such as the GitHub page, guides and website, and a suite of networking-focused development and management tools, including a custom web-based integrated development environment. \\

The developed networking approach and platform were tested using various methods, including robustness tests, experiments with range and received signal strength indicators, and two hackathons with college students to assess their utilisation of the capabilities provided and the application concepts they devised. The developed networking approach has also been adopted by students and incorporated  into other ongoing research projects, such as the Smart Playground project. The outcomes of this research demonstrate the viability of the peer-to-peer communication system and its potential for educational applications. However, the research also highlights shortcomings in the usability and accessibility of the support tools developed. The research further proposes areas for future development, such as improving the quality and accessibility of support tools and materials. This work contributes to the field of educational robotics by providing a flexible, accessible networking solution and support platform that can be built upon to enhance interactive learning environments. The integration of peer-to-peer communication between educational robotics modules and the facilitating Smart System Platform can support and enable novel innovative approaches to STEM education, as well as the development of more complex, collaborative and engaging learning experiences.

