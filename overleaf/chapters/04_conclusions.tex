\cleardoublepage%
\chapter{\label{chap:con}Conclusions}%
%1-2 pages
%This section does not need to be long, but it needs to be concrete. Explain what you found, what you did not find, and what needs to be done next. No more than two pages maximum. 

This thesis builds on the Smart Motors project \citep{dahal_designing_2024} by developing a peer-to-peer networking approach based on ESP-NOW for communication between different MCs. To support this networking capability, a framework, the Smart System Platform, was designed around the networking capability to improve accessibility and usability. Based on this design, an MVP of the platform components was developed to ensure that networking between modules was not only possible, but also practical and easy to implement. 
In addition to developing and testing the networking library, this thesis also investigated how the platform could be used in practice. Various components of the Smart System Platform were designed and tested, and their usability was evaluated through trials with students at the College of Education. This provided valuable insights into how educators and students interact with the system, how they might use it, and what kind of concepts they might come up with. \\\\

The results of the research confirm that the networking approach developed is both viable and functional, demonstrating great potential for practical applications. The peer-to-peer communication system works reliably both in controlled test environments and in independent projects by different users. The networking framework successfully enables direct communication between smart modules, making it feasible for a wide range of applications, from collaborative robotics to interactive educational experiences.
Although still in its early stages, the Smart System Platform provides a functional proof of concept of how networked educational robotics can be structured. While the Smart System Platform aimed to provide an accessible means for users to engage with the networking capabilities, some usability challenges were identified. The MVP successfully demonstrated the feasibility of the networking system and the platform-based approach, but further refinements are needed to improve the accessibility, usability and intuitiveness of the platform's supporting material. In particular, the accessibility and usability of the supporting tools and materials pose challenges, particularly the initial learning curve and accessibility of the supporting tools, although the core concept remains robust. \\\\

One of the key findings of this thesis is the potential of networked educational robotics to enhance interactive learning environments. Educational use cases, such as the Smart Playground project \citep{jess_smart_2025}, provide an example of how the Smart System Platform and its networking capabilities can be used to support the development of creative and engaging educational experiences. The results of the hackathons further highlight the innovative concepts and ideas that emerge when students and educators are given access to this technology. The range of concepts developed in hackathons and external projects reinforces the viability of the approach and demonstrates the creative potential of networked modules in education and research. \\\\

This thesis demonstrates that peer-to-peer networking for the so-called smart modules is both feasible and valuable, particularly in the context of education and robotics. While challenges remain, the potential for this technology is enormous. By refining the tools, improving accessibility and fostering a community of users and developers, the Smart System Platform can become a transformative tool for networked robotics, STEM education and interactive learning.
By continuing to develop and enhance this foundation, the Smart System Platform has the potential to become a widely used and highly impactful educational technology, enabling students, educators and researchers to explore new frontiers in connected robotics and collaborative learning.






