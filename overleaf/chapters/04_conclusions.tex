\cleardoublepage%
\chapter{\label{chap:con}Conclusions}%
%1-2 pages
%This section does not need to be long, but it needs to be concrete. Explain what you found, what you did not find, and what needs to be done next. No more than two pages maximum. 

The research in this thesis builds on the Smart Motors project \citep{dahal_designing_2024} by developing a peer-to-peer networking approach using ESP-NOW for communication between different MCs. To support this networking capability, the Smart System Platform framework was designed to improve accessibility and usability. A minimum viable product (MVP) of the platform and its components was developed to ensure that networking between modules was not only possible, but also practical and easy to implement. 
In addition to developing and testing the networking library, way how the platform could be used in practice were also investigated. Various components of the Smart System Platform were designed and tested, and their usability was evaluated through trials with college students. These evaluations provided valuable insights into how educators and students interact with the system, how they might use it, and what kinds of concepts they might develop.\\

The research findings demonstrate that the developed networking approach is both viable and functional, thus highlighting its significant potential for practical applications. The peer-to-peer communication system functions reliably in both controlled test environments and independent projects by different users. The networking framework successfully enables direct communication between Smart Modules, thus rendering it suitable for a wide range of applications, from collaborative robotics to interactive educational experiences.
Despite being in its early stages, the Smart System Platform serves as a functional proof of concept, demonstrating the potential of networked educational robotics and a structured approach to implementation. While the Smart System Platform was designed to provide an accessible means for users to engage with the networking capabilities, some usability challenges were identified. The MVP successfully demonstrated the feasibility of both the networking system and the platform-based approach. However, further refinements are required to enhance the accessibility, usability and intuitiveness of the platform's supporting material. The accessibility and usability of the supporting tools and materials pose challenges, particularly due to the initial learning curve, though the core concept remains robust. \\

A key finding of this thesis is that networked educational robotics could be viably used, from a technological perspective, for interactive and collaborative learning approaches. Educational use cases, such as the Smart Playground project \citep{blake-west_smart_2025}, provide an example of how the Smart System Platform and its networking capabilities can be used to support and facilitate the development of creative and engaging educational experiences. The results of the hackathons further highlight the innovative concepts and ideas that emerge when students and educators are given access to this technology. The range of concepts developed in hackathons and external projects reinforces the viability of the approach and to demonstrate the creative potential of networked modules in education and research. \\

Peer-to-peer networking for the Smart Modules is both feasible and valuable, particularly in the context of education and robotics. Despite remaining challenges, this technology has great potential. By refining the tools, improving accessibility and fostering a community of users and developers, the Smart System Platform could become a transformative tool for networked robotics, STEM education and interactive learning.
With further development and refinement, the Smart System Platform could become a highly effective educational tool, enabling students, educators, and researchers to explore new frontiers in connected robotics and collaborative learning.






