\cleardoublepage%
\chapter{\label{chap:methods}Methods}%

This is arguably the most important section of your work.  In the methods section you need to explain exactly what you did, how you did it and why you did it. Describe the work you did with enough detail that someone could replicate your work exactly.

Below are some common subheadings that are used, but the exact list will depend on your specific project design and method.

The research in this thesis can be split into three distinct phases, phase 1 consist of the status quo in the field, analysis and the platform architecture and design, as well as development of the framework and guiding principles. Phase 2 encompasses the development of the base of the Smart System Platform and the development of key capabilities. Phase 3 is the validation and active use of the designed platform to develop projects and educational experiences.

\section{\label{sec:methods_platform_dev}Platform Architecture Development and Approach}

Iterative development of both the platform and the other stuff

\subsection{\label{sec:methods_sm_analysis}Smart Motor Project and Analysis}

\subsection{\label{sec:methods_tech_review}Technology and Literature Review}

Defined the idea, structure, goals, framework, key necessities and guiding principles and why. 
Focus on capabilities and accessibility for educational project / lesson development. 

\subsection{\label{sec:methods_exp1}Platform Architecture and Design Development}

\subsection{\label{sec:methods_nomenclature}Nomenclature}

Design of the Platform, what parts does it need, go into detail about the different parts of the platform, identify the aspects, key capabilities, hosting, guides and whatever. 

What aspects are important? Hardware and Software of the smart system, mention all the networking capabilities etc. Module hardware ideas. Mention the many hardware components that could be used! Modularity!

IDE for easy development using smart system modules and interactions as well as lessons. Mention the difficulty of hard coding and how it is great to do learning by doing, but 

Key stakeholders, who is it for?

Website, accessibility, information. Free to use, all info publicly available, hosted on GitHub, including website and development tools and guides.

Also mention the development of similar projects and testing od educational experiences, which are often done from scratch. Allow the use of the smart system platform as an unrestricted platform to develop own lessons, projects and experiences on, using existing capabilities. 

\section{\label{sec:methods_sspd}Smart System Platform Development}

\subsection{\label{sec:methods_gh}GitHub}
Layout, access, automation, structure, open data/science, license.

\subsection{\label{sec:methods_website}Website}
Outreach and accessibility.
All parts of the website, guides, introduction, other relevant information. 

\subsection{\label{sec:methods_ide}Integrated Development Environment}
Usability and accessibility
pyscript, ui and whatnot. 


\subsection{\label{sec:methods_hw}Hardware}
Modules, overview of existing technology, go into the analysis results of the used software, firmware and chips, how modular they are and what could be used for them and show the graphics with example input / output matrix.

\subsection{\label{sec:methods_sw}Software}


\subsection{\label{sec:methods_fw}Firmware}
Mention that it was considered to develop own firmware with deeloped capabilities, but consideration of ease of reproducability and use against high customisation of used firmware, yet a little customisation was needed.


\section{\label{sec:methods_exp_p3}Validation and Testing (Phase 3)}
Validation, us of the platform, its tools and capabilities.
Mention the use of capabilities in other projects, such as the playground project, by some students for own little projects and stuff.
Hackathon
Example Kit and Projects (Music box, 


